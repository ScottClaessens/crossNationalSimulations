\newpage

\section*{Supplementary Information}

\subsection*{Data-generating model for simulations}

Following the approach in our initial article (Claessens et al., 2023), we 
simulated data for 150 nations $i$ with varying degrees of religious 
autocorrelation for outcome $y$ and predictor $x$ using the following
generative model:

\begin{align}
\begin{bmatrix}y_i\\x_i \end{bmatrix} &\sim \text{MVNormal}
\begin{pmatrix}\begin{bmatrix}\alpha_y\\\alpha_x \end{bmatrix},
\textbf{S}\end{pmatrix} \nonumber \\
\alpha_y &\sim \text{Normal}(0, \sqrt{\lambda} \cdot \Sigma_\text{Dow}) 
\nonumber \\
\alpha_x &\sim \text{Normal}(0, \sqrt{\rho} \cdot \Sigma_\text{Dow}) 
\nonumber \\
\textbf{S} &= 
\begin{pmatrix}\sqrt{1 - \lambda} & 0 \\ 0 & \sqrt{1 - \rho} \end{pmatrix}
\begin{pmatrix}1 & r \\ r & 1\end{pmatrix}
\begin{pmatrix}\sqrt{1 - \lambda} & 0 \\ 0 & \sqrt{1 - \rho} \end{pmatrix} 
\nonumber
\end{align}

In this generative model, $\Sigma_\text{Dow}$ is the national-level religious
proximity matrix that Wolfer and Koplenig (2024) constructed using data from Dow
and Karunaratna (2006). In addition, $\lambda$ and $\rho$ are autocorrelation 
parameters that represent the expected religious "signal" for outcome and 
predictor variables, respectively, and $r$ is the true cross-national 
correlation between the variables after accounting for religious
autocorrelation. We set $\lambda$ and $\rho$ to 0.2 (weak), 0.5 (moderate), or 
0.8 (strong). For simplicity, we always set $r = 0$ in this simulation. For each
parameter combination, we simulated 100 datasets, resulting in 900 datasets.

\subsection*{Analysis without controls}

We first analysed the simulated data without any control variables. The
statistical model is as follows:

\begin{align}
y_i &\sim \text{Normal}(\mu_i, \sigma) \nonumber \\
\mu_i &= \alpha + \beta x_i \nonumber \\
\alpha &\sim \text{Normal}(0, 0.5) \nonumber \\
\beta &\sim \text{Normal}(0, 0.5) \nonumber \\
\sigma &\sim \text{Exponential}(5) \nonumber
\end{align}

We fitted each model using the \textit{brms} package (Bürkner, 2017) with four 
chains, 1000 warmup samples, and 1000 post-warmup samples. We calculated the 
false positive rate as the proportion of $\beta$ slopes with 95\% credible 
intervals excluding zero.

\subsection*{Analysis using alternative religious proximity matrix}

For our next analysis, we used a religious proximity matrix of countries from 
Kyritsis et al. (2022). This matrix is based on national-level data on religious
adherence from the Association of Religion Data Archives (ARDA) together with a
family tree representing genealogical relationships between 28 religious
lineages, informed by historical sources. Since there is evidence for horizontal
transmission between some lineages, eight different cladograms were considered,
representing alternative paths of inheritance between lineages. For further
details on the construction of these religious family trees, see Kyritsis et al.
(2022).

For each of the eight cladograms, a pairwise distance matrix was generated based
on patristic distances between religious traditions. These distances were then 
converted to proximities and weighted by adherent percentages from ARDA 
using the following formula (Eff, 2008):

\[R_{rk} = \sum_{r} \sum_{k} p_{ik} p_{jr} s_{ij}\]

where $R_{rk}$ is the religious connection between countries $r$ and $k$,
$p_{ik}$ is the percentage of the population in country $k$ adhering to religion
$i$, $p_{jr}$ is the percentage of the population in country $r$ adhering to 
religion $j$, and $s_{ij}$ is the religious proximity measure between religions
$i$ and $j$. The resulting eight matrices were averaged to produce the final
matrix, which we label $\Sigma_\text{ARDA}$.

We then included this $\Sigma_\text{ARDA}$ matrix in the statistical model 
above, allowing nation random effects to covary according to alternative
religious proximities. The statistical model is as follows:

\begin{align}
y_{i} &\sim \text{Normal}(\mu_{i},\sigma) \nonumber \\
\mu_{i} &= \alpha + z_{\text{NATION}[i]}\sigma_{\alpha}\Sigma_\text{ARDA} + 
\beta x_{i} \nonumber \\
\alpha &\sim \text{Normal}(0, 0.5) \nonumber \\
\beta &\sim \text{Normal}(0, 0.5) \nonumber \\
z_{j} &\sim \text{Normal}(0, 1) \nonumber \\
\sigma_{\alpha} &\sim \text{Exponential}(5) \nonumber \\
\sigma &\sim \text{Exponential}(5) \nonumber
\end{align}

We fitted each model using the \textit{brms} package (Bürkner, 2017) with four 
chains, 1000 warmup samples, and 1000 post-warmup samples. We calculated the 
false positive rate as the proportion of $\beta$ slopes with 95\% credible 
intervals excluding zero.

\subsection*{Description of code functionality}

The R code in our GitHub repository uses the \textit{targets} package
(Landau, 2021) to create the analysis pipeline for this manuscript. The code 
loads the proximity matrices, calculates the correlations between these
matrices, runs the simulations, plots the results of the simulations, and
reproducibly generates the manuscript file.

\subsection*{Supplementary References}

Bürkner, P. brms: An R package for Bayesian multilevel models using Stan. 
\textit{Journal of Statistical Software} \textbf{80}, 1-28 (2017).

Claessens, S., Kyritsis, T. \& Atkinson, Q. D. Cross-national analyses require 
additional controls to account for the non-independence of nations.
\textit{Nature Communications} \textbf{14}, 5776 (2023).

Dow, D. \& Karunaratna, A. Developing a multidimensional instrument to measure 
psychic distance stimuli. \textit{Journal of International Business Studies}
\textbf{37}, 578-602 (2006).

Eff, E. A. Weight matrices for cultural proximity: Deriving weights from a 
language phylogeny. \textit{Structure and Dynamics} \textbf{3} (2008).

Kyritsis, T., Matthews, L. J., Welch, D. \& Atkinson, Q. D. Shared cultural 
ancestry predicts the global diffusion of democracy. \textit{Evolutionary Human 
Sciences} \textbf{4}, e42 (2022).

Landau, W. M. The targets R package: a dynamic Make-like function-oriented
pipeline toolkit for reproducibility and high-performance computing.
\text{Journal of Open Source Software} \textbf{6}, 2959 (2021).

Wolfer, S. \& Koplenig, A. Leakage explains the apparent superiority of Bayesian
random effect models - a preregistered comment on Claessens, Kyritsis and 
Atkinson (2023). (2024) doi:10.31219/osf.io/ex267.
